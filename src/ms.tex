% Define document class
\documentclass[twocolumn]{aastex631}

% Filler text
\usepackage{blindtext}

% Begin!
\begin{document}

% Title
\title{An open source scientific article}

% Author list
\author{First Author}

% Abstract with filler text
\begin{abstract}
    \blindtext
\end{abstract}

% Main body
\section{Introduction}

Figure~\ref{fig:mandelbrot:original} and Figure~\ref{fig:mandelbrot:red} were
generated from the same figure script. Normally in \texttt{showyourwork}, 
figures are labeled according to the script that generated them, but since the same
script generated the two figures, that would result in duplicated labels.
The way around this is to append a colon followed by a unique tag to each figure label.
The value of the tag is ignored internally---it serves only to differentiate between
the two figure labels on the LaTeX side.

% A sample figure
\begin{figure}[ht!]
    \begin{centering}
        \includegraphics[width=0.4\linewidth]{figures/mandelbrot.pdf}
        \caption{
            This figure was generated by the script \texttt{mandelbrot.py}
            and is labeled \texttt{fig:mandelbrot:original}.
        }
        % This label tells showyourwork that the script `figures/mandelbrot.py'
        % generates the PDF file included above
        \label{fig:mandelbrot:original}
    \end{centering}
\end{figure}

\begin{figure}[ht!]
    \begin{centering}
        \includegraphics[width=0.4\linewidth]{figures/mandelbrot_red.pdf}
        \caption{
            This figure was generated by the script \texttt{mandelbrot.py}
            and is labeled \texttt{fig:mandelbrot:red}.
        }
        % This label tells showyourwork that the script `figures/mandelbrot.py'
        % generates the PDF file included above
        \label{fig:mandelbrot:red}
    \end{centering}
\end{figure}

\end{document}
